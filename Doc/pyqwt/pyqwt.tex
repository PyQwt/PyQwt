\documentclass{manual}

% Links
\newcommand{\me}{\ulink{me}{mailto:gvermeul@grenoble.cnrs.fr}}
\newcommand{\Qwt}{\ulink{Qwt}{http://qwt.sourceforge.net}}

% Source code
\newcommand{\SipGpl}{\ulink{sip-x11-gpl-3.8.tar.gz}{http://www.river-bank.demon.co.uk/download/sip/sip-x11-gpl-3.8.tar.gz}}
\newcommand{\SipNcZip}{\ulink{sip-win-nc-3.8.zip}{http://www.river-bank.demon.co.uk/download/sip/sip-win-nc-3.8.zip}}
\newcommand{\PyQtGpl}{\ulink{PyQt-x11-gpl-3.8.1.tar.gz}{http://www.river-bank.demon.co.uk/download/PyQt/PyQt-x11-gpl-3.8.1.tar.gz}}
\newcommand{\PyQtNcZip}{\ulink{PyQt-win-nc-3.8.zip}{http://www.river-bank.demon.co.uk/download/PyQt/PyQt-win-nc-3.8.zip}}
\newcommand{\snapshot}{\ulink{snapshot}{http://www.river-bank.demon.co.uk/download/snapshots}}
\newcommand{\NumericTarGz}{\ulink{Numeric-23.1.tar.gz}{http://prdownloads.sourceforge.net/numpy/Numeric-23.0.tar.gz?download}}
\newcommand{\numarrayTarGz}{\ulink{numarray-0.7.tar.gz}{http://prdownloads.sourceforge.net/numpy/numarray-0.7.tar.gz?download}}
\newcommand{\PyQwtTarGz}{\ulink{PyQwt-3.8.tar.gz}{http://prdownloads.sourceforge.net/pyqwt/PyQwt-3.8.tar.gz?download}}
\newcommand{\PyQwtTarGzOld}{\ulink{PyQwt-3.7.tar.gz}{http://prdownloads.sourceforge.net/pyqwt/PyQwt-3.7.tar.gz?download}}
\newcommand{\ReadMePatches}{\ulink{}{http://pyqwt.sourceforge.net/patches/README}}

% Installers for MS-Windows
\newcommand{\PythonExeBB}{\ulink{Python-2.2.3.exe}{http://www.python.org/ftp/python/2.2.3/Python-2.2.3.exe}}
\newcommand{\PythonExeBC}{\ulink{Python-2.3.exe}{http://www.python.org/ftp/python/2.3/Python-2.3.exe}}
\newcommand{\NumericExeBB}{\ulink{Numeric-23.0.win32-py2.2.exe}{http://prdownloads.sourceforge.net/numpy/Numeric-23.0.win32-py2.2.exe?download}}
\newcommand{\NumericExeBC}{\ulink{Numeric-23.1.win32-py2.3.exe}{http://prdownloads.sourceforge.net/numpy/Numeric-23.1.win32-py2.3.exe?download}}
\newcommand{\NumArrayExeBB}{\ulink{numarray-0.7.win32-py2.2.exe}{http://prdownloads.sourceforge.net/numpy/numarray-0.7.win32-py2.2.exe?download}}
\newcommand{\NumArrayExeBC}{\ulink{numarray-0.7.win32-py2.3.exe}{http://prdownloads.sourceforge.net/numpy/numarray-0.7.win32-py2.3.exe?download}}
\newcommand{\QtNcExe}{\ulink{QtWin230-NonCommercial.exe}{ftp://ftp.trolltech.com/qt/non-commercial/QtWin230-NonCommercial.exe}}
\newcommand{\PyQtExeBB}{\ulink{PyQt-win-nc-3.7.exe}{http://pyqwt.sourceforge.net/support/PyQt-win-nc-3.7.exe}}
\newcommand{\PyQtExeBC}{\ulink{PyQt-win-nc-3.8.exe}{http://www.river-bank.demon.co.uk/download/PyQt/PyQt-win-nc-3.8.exe}}
\newcommand{\PyQtEduExe}{\ulink{PyQt-win-edu-msvc-3.8.1.exe}{http://www.riverbankcomputing.co.uk/download/PyQt-win-edu-msvc-3.8.1.exe}}
\newcommand{\PyQwtExeBB}{\ulink{PyQwt\_Qt230-3.7.win32-py2.2.exe}{http://prdownloads.sourceforge.net/pyqwt/PyQwt\_Qt230-3.7.win32-py2.2.exe?download}}
\newcommand{\PyQwtExeBC}{\ulink{PyQwt\_Qt230-3.8.win32-py2.3.exe}{http://prdownloads.sourceforge.net/pyqwt/PyQwt\_Qt230-3.8.win32-py2.3.exe?download}}

\title{PyQwt Manual}

% boilerplate.tex?
\author{Gerard Vermeulen}

\date{\today}
\release{3.8}
%\setreleaseinfo{cvs}
\setshortversion{3.8}

\makeindex

\begin{document}

\maketitle

% This makes the contents more accessible from the front page of the HTML.
\ifhtml
\chapter*{Front Matter \label{front}}
\fi

Copyright \copyright{} 2001-2003 Gerard Vermeulen

Copyright \copyright{} 2000 Mark Colclough

PyQwt is free software; you can redistribute it and/or modify it under the
terms of the GNU General Public License as published by the Free Software
Foundation; either version 2 of the License, or (at your option) any later
version.

PyQwt is distributed in the hope that it will be useful, but WITHOUT ANY
WARRANTY; without even the implied warranty of MERCHANTABILITY or FITNESS
FOR A PARTICULAR PURPOSE.  See the GNU  General Public License for more
details.

You should have received a copy of the GNU General Public License along with
PyQwt; if not, write to the Free Software Foundation, Inc., 59 Temple Place,
Suite 330, Boston, MA 02111-1307, USA.

In addition, as a special exception, Gerard Vermeulen gives permission to
link PyQwt with commercial, non-commercial or educational versions of Qt,
PyQt and sip, and distribute PyQwt in this form, provided that equally
powerful versions of Qt, PyQt and sip have been released under the terms
of the GNU General Public License.

If PyQwt is dynamically linked with commercial, non-commercial or
educational versions of Qt, PyQt and sip, PyQwt becomes a free plug-in
for a non-free program.



\begin{abstract}

\noindent
PyQwt is a set of Python bindings for the \Qwt{} library featuring fast
plotting of lists, tuples and arrays provided by the Python extensions
\module{Numeric} and \module{numarray}.

\end{abstract}

\tableofcontents

\chapter{Introduction \label{introduction}}

PyQwt is a set of Python bindings for the \Qwt{} library featuring fast
plotting of lists, tuples and arrays which are provided by the Python
extensions \module{Numeric} and \module{numarray}.

\section{Qwt \label{qwt-intro}}

Qwt is a C++ library based on the Qt GUI framework from Troll Tech AS.
The Qwt library contains widgets and components useful to write technical
and scientific programes.
It includes the following widgets:
\begin{itemize}
  \item
    \class{QwtCompass}: a very fancy QDial-like widget to display and
    control a direction.
  \item
    \class{QwtCounter}: a QSpinBox-like widget to display and control
    a bounded floating point value.
  \item
    \class{QwtDial}: a QDial-like widget to display and control a
    floating point value.
  \item
    \class{QwtKnob}: a potentiometer-like widget to display and control
    a bounded floating point value.
  \item
    \class{QwtPlot}: a widget to plot two-dimensional data points.
  \item
    \class{QwtSlider}: a QSlider-like widget to display and control a
    bounded floating point value.
  \item
    \class{QwtThermo}: a thermometer-like widget to display a floating
    point value. 
  \item
    \class{QwtWheel}: a wheel-like widget with its axis parallel to
    the computer screen to control a floating point value over a very
    large range in very small steps.
\end{itemize}
The widgets are documented in the
\ulink{Qwt manual}{qwt/index.html}
on your hard disk or \Qwt{} on the web.


\section{Numeric \label{numeric-intro}}

The \ulink{Numerical Python Extensions}{http://www.numpy.org}
(also known as Numeric or NumPy) turn Python in an ideal language for
experimental numerical and scientific computing (MatLab- or IDL-like,
but Python offers a superior programming language).
NumPy defines a new data type "array" and a very complete of operators and
functions to manipulate NumPy arrays.
Look at the following simple example:

\verbatiminput{numeric.example}

The function call \samp{arange(0.0, 10.0, 3.0)} returns a NumPy array of 4
equidistant points from 0 to 9 inclusive: \samp{array([ 0., 3., 6., 9.])}.
The function call \samp{sin(x)} and statement \samp{x*x} show that
\samp{sin} and \samp{*} manipulate NumPy arrays element by element.
All this in C, for a manifold speedup with respect to pure Python.


\section{PyQwt \label{pyqwt-intro}}

The PyQwt widgets can be used in Graphical User Interface (GUI)
programs as well as from the Command Line Interpreter (CLI).
This section addresses the CLI facilities.

The documentation for writing GUI programs consists of the
\ulink{Qwt manual}{qwt/index.html} on your hard disk or the \Qwt{} manual on
the web, supplemented by chapter \ref{reference} to document the differences
between PyQwt and Qwt.

\subsection{PyCute \label{pycute-intro}}

\program{PyCute} embeds a Python shell in a \class{QMultiLineEdit} widget
(Qt-2.x) or a \class{QTextEdit} widget (Qt-3.x) that works nicely with
\module{qwt.qplt}, see \ref{qwt-qplt}.

Possible invokations of the \program{PyCute} are:
\begin{verbatim}
PyCute --demo
\end{verbatim}
or:
\begin{verbatim}
PyCute
\end{verbatim}
or from the examples directory:
\begin{verbatim}
PyCute ICompass.py
\end{verbatim}

The first command creates a PyCute shell and executes eight Python statements
to create a \class{Plot} and an \class{IPlot} widget.

Both widgets allow you to select a new zoom region by pressing the left button,
dragging the mouse and releasing the left button.
The previous zoom region is placed on an infinite stack, so that you can
unzoom to previous zoom regions by clicking the right button.

Both widgets allow you to hide/unhide curves by clicking on a legend item.

In addition, the \class{IPlot} widgets displays the plot coordinates in the
status bar and has a toolbar buttons to print and to clone the plot into
\ulink{Grace}{http://plasma-gate.weizmann.ac.il/Grace/}.
Grace is a WYSIWYG plot program that produces output very well adapted to
scientific journals.

The second command gives you a plain PyCute shell and the third command 
interpretes the ICompass.py program.

Type \kbd{Control-D} to exit the shell.
 

\subsection{iqt \label{iqt-intro}}

The programming style \emph{imposed} by PyQt (and other GUI toolkit wrappers)
is to create an application instance with some widgets and to enter into an
event loop that locks you from the Python command line interpreter.

Module \module{iqt} enables you to:
\begin{itemize}
\item
  create and control (Py)Qt widgets from the command line interpreter or an
  enhanced interpreter like \ulink{ipython}{http://ipython.scipy.org}.
\item
  to interact with those widgets through (Py)Qt's GUI.
\end{itemize}

Module \module{iqt} works by hooking \cfunction{qApp->processEvents()} on the
event hook of the readline module.
The GNU readline library closes the event loop by calling 
\cfunction{qApp->processEvent()} at a maximum rate of 10 times per second
while reading your keyboard input.

To see how \module{iqt} works, play with \file{ICompass.py} by running
\begin{verbatim}
python -i ICompass.py
\end{verbatim}
or
\begin{verbatim}
python ICompass.py
\end{verbatim}
or
\begin{verbatim}
ipython ICompass.py
\end{verbatim}

\begin{notice}[warning]
  Applying the builtin function \function{dir()} on classes made by sip-3.6
  and earlier does only return names of attributes and methods that have
  been used before (due to a lazy import mechanism).

  This behavior impairs the usefulness of the module \module{rlcompleter}.
  The classes in PyQwt offer a method \method{getLazyNames()} as an
  alternative to be used in combination with the modules
  \module{iqt.rlcompleter} and \module{iqt.FlexCompleter} instead of
  \module{rlcompleter} in the standard Python library and
  \module{IPython.FlexCompleter}.

  Full support for the builtin function \function{dir()} is present in
  sip-3.7; in this case the standard readline completers will work. 
\end{notice}

\begin{notice}[warning]
  The module \module{iqt} will not work with a MSVC-compiled Python, because
  it has no \module{readline} module.
\end{notice}


\subsection{qwt.qplt \label{qwt-qplt}}

The module \module{qwt.qplt} defines a tiny data-directed programming
language to simplify the use of \class{QwtPlot} from the Python command
line interpreter or from an embedded Python shell like in \program{PyCute}.
Here follows a small example:

\verbatiminput{qplt.example}

\subsubsection{qwt.qplt.Plot \label{intro-qplt-plot}}

\begin{classdesc}{Plot}{*arguments}
  \class{Plot} takes any number of optional arguments.
  The data-directed interpretation of each optional argument is:
  \begin{itemize}
  \item
    \class{Axis}, to enable and redefine an axis.
  \item
    \class{Curve}, to add a curve to the plot.
  \item
    \class{string} or a \class{QString}, to set the title of the plot.
  \item
    (\class{int}, \class{int}), to set the size of the plot.
  \end{itemize}

  \class{Plot} has an infinite deep stack of zoom regions.
  You can select a new zoom region by pressing the left mouse button
  inside the plot, dragging the mouse and releasing the button.
  The old zoom region is pushed on the stack.
  On each right-click inside the plot, an old zoom region is
  popped from the stack.

  \class{Plot} uses \method{__getattr__()}, \function{hasattr()} and
  \function{getattr()} to inherit all methods of \class{QwtPlot}.
\end{classdesc}

In addition, the class \class{Plot} defines the following methods:

\begin{methoddesc}[Plot]{formatCoordinates}{self, x, y}
Formats mouse coordinates into (real world) plot coordinates.
\end{methoddesc}

\begin{methoddesc}[Plot]{gracePlot}{self}
Clones the plot to display it in Grace for very high quality output.
\end{methoddesc}

\begin{methoddesc}[Plot]{plot}{self, *curves}
Plots a list of curves.
\end{methoddesc}

\begin{methoddesc}[Plot]{plotAxis}{self, orientation, options, title}
Plots and enables an axis.
\end{methoddesc}

\begin{methoddesc}[Plot]{plotCurve}{self, curve}
Plots a single curve.
\end{methoddesc}

\begin{methoddesc}[Plot]{onMouseMoved}{self, e}
Is a SLOT for SIGNAL('plotMouseMoved(const QMouseEvent\&)').
\end{methoddesc}

\begin{methoddesc}[Plot]{onMousePressed}{self, e}
Is a SLOT for SIGNAL('plotMousePressed(const QMouseEvent\&)'). 
\end{methoddesc}

\begin{methoddesc}[Plot]{onMouseReleased}{self, e}
Is a SLOT for SIGNAL('plotMouseReleased(const QMouseEvent\&)').
\end{methoddesc}

\begin{methoddesc}[Plot]{toggleCurve}{self, key}
Is a SLOT for SIGNAL('legendClicked(long)') to dis/enable a curve.
\end{methoddesc}

\subsubsection{qwt.qplt.IPlot \label{intro-qplt-iplot}}

\begin{classdesc}{IPlot}{*arguments}
  \class{IPlot} is a main window with a Plot widget as central widget.

  The data-directed interpretation of each optional argument is the
  same as for \class{Plot}.

  The functionality of \class{Plot} is enhanced with a QToolBar with
  buttons for printing and cloning into grace.
  In addition, the position of the cursor in the plot region is tracked
  and displayed in a status bar. 
\end{classdesc}

\subsubsection{qwt.qplt.Curve \label{intro-qplt-curve}}

\begin{classdesc}{Curve}{x, y, *arguments}
  \class{Curve} takes two obligatory arguments followed by any number of
  optional arguments. The arguments \var{x} and \var{y} must be sequences
  of \class{float} objects.
  The data-directed interpretation of each optional argument is:
  \begin{itemize}
  \item
    \class{Axis}, to attach an axis to the curve.
  \item
    \class{Pen}, to select a pen to connect the data points.
  \item
    \class{Symbol}, to select a symbol to the data points.
  \item
    \class{string} or \class{QString}, to set the title of the curve.
  \end{itemize}
\end{classdesc}

\subsubsection{qwt.qplt.Axis \label{intro-qplt-axis}}

\begin{classdesc}{Axis}{*arguments}
  \class{Axis} takes any number of optional arguments.
  The data-directed interpretation of each optional argument is:
  \begin{itemize}
  \item
    \class{AxisOrientation}, to set the orientation of the axis.  Predefined
    instances of \class{AxisOrientation} are \constant{Left}, \constant{Right},
    \constant{Bottom} and \constant{Top}.
  \item
    \class{int}, to set the options of the axis.
  \item
    \class{string} or \class{QString}, to set the title of the axis.
  \end{itemize}
\end{classdesc}

\subsubsection{qwt.qplt.Pen \label{intro-qplt-pen}}

\begin{classdesc}{Pen}{*arguments}
  \class{Pen} takes any number of optional arguments.
  The data-directed interpretation of each optional argument is:
  \begin{itemize}
  \item
    \class{PenStyle}, to set the style of the pen.
  \item
    \class{QColor}, to set the color of the pen.
  \item
    \class{int}, to set the width of the pen.
  \end{itemize}
\end{classdesc}

\subsubsection{qwt.qplt.Symbol \label{intro-qplt-symbol}}

\begin{classdesc}{Symbol}{*arguments}
  \class{Symbol} takes any number of optional arguments.
  The data-directed interpretation of each optional argument is:
  \begin{itemize}
  \item
    \class{QColor}, to set the fill color of the symbol.
  \item
    \class{SymbolStyle}, to set the style of the symbol. Predefined instances
    of \class{SymbolStyle} are \constant{NoSymbol}, \constant{Circle},
    \constant{Square} and \constant{Diamond}.
  \item
    \class{int}, to set the size of the symbol.
  \end{itemize}
\end{classdesc}

\chapter{Installation \label{installation}}

You can install PyQwt on \POSIX{} and Microsoft Windows operating systems:
\begin{enumerate}
  \item
    see \ref{win-binary-install} for instructions on the use of the binary
    Windows installer, to be used with the non-commercial releases of
    Qt and PyQt.
  \item
    see \ref{posix-source-install} for installation instructions starting
    from source code for Unix and Linux.
  \item
    see \ref{win-source-install} for installation instructions starting from
    source code for Windows, MSVC and Qt-Non-Commercial.
    Thanks to feedback from developers with a commercial Qt license the
    installation scripts have been adapted to Qt-Educational, Qt-3.0.x,
    Qt-3.1.x and Qt-3.2.0.
    Contact \me, if you run into problems.
\end{enumerate}
\begin{notice}[note]
The compatibility of PyQwt's build scripts with the educational and commercial
packages of Qt and PyQt depends on \strong{your} feedback, because I do not
have access to those versions of Qt and PyQt.
\end{notice}

\section{Binary Installation on Windows \label{win-binary-install}}

First, you have to install Python, Numeric, Qt-Non-Commercial and
PyQt-Non-Commercial by executing the following installation programs:
\begin{enumerate}
  \item
    \PythonExeBC{}
  \item
    \NumericExeBC{}
  \item
    \QtNcExe{}
  \item
    \PyQtExeBC{}
\end{enumerate}
Then, run \PyQwtExeBC{} and you are set.
Example programs and documentation for PyQwt and \Qwt{} are installed in the
folder \file{C:/Python23/PyQwt}.

\begin{notice}[warning]
This documentation is for the future PyQwt-3.8. If your only choice is
\PyQwtExeBB, you will need to install \PythonExeBB{}, \NumericExeBB{},
\QtNcExe{} and \PyQtExeBB{}.
\end{notice}


\section{Source installation on \POSIX{} \label{posix-source-install}}

\subsection{Prerequisites for building PyQwt on \POSIX{}
  \label{posix-prereq}}

Recent versions of the following software should be installed on your system:
\begin{enumerate}
  \item
    \ulink{Python}{http://www.python.org}.
    I am using Python-2.3 and Python-2.2.x, but Python-2.1.x and probably
    Python-2.0.x should also work.
    Earlier versions of Python are not supported.
  \item
    \ulink{Qt}{http://www.trolltech.com}.
    I am using Qt-3.2.x, Qt-3.1.x, Qt-3.0.x, Qt-2.3.1 and Qt-2.3.0.
    Earlier versions of Qt are not supported.
  \item
    \SipGpl{}.
    I support sip downto version 3.4, but the most recent version
    gets most testing. You may also try a \snapshot at your own risk.
  \item
    \PyQtGpl{}.
    I support PyQt downto version 3.4, but the most recent version
    gets most testing. You may also try a \snapshot at your own risk.
  \item
    Read \ReadMePatches{} to check out bug fixes and upgrades.
\end{enumerate}

To exploit the full power of PyQwt's plot widget, you should install at
least one the Numerical Python extensions:
\ulink{Numeric}{http://www.numpy.org}
or its successor
\ulink{numarray}{http://www.stsci.edu/resources/software_hardware/numarray}.
I am using \NumericTarGz{} and \numarrayTarGz{}. 
Versions of Numeric later than 21.0 are supported.
Numarray is newer and therefore less stable than Numeric, so get the latest!

The source package \PyQwtTarGz{} contains a CVS version of the \Qwt{} library.

\begin{notice}[warning]
Always try to install sip and PyQt with matching version numbers: only
\code{micro} may differ for version numbers of the form
\code{major.minor.micro}.
\end{notice}

\subsection{Build on \POSIX{} \label{posix-build}}

\begin{notice}[warning]
This documentation is for the future PyQwt-3.8. If \PyQwtTarGz{} does not
exist, fall back on \PyQwtTarGzOld{} and read the \ReadMePatches{} to build
PyQwt-3.7 for PyQt\&sip-3.8 or -3.8.1.
\end{notice}

\begin{enumerate}
\item
  Unpack \PyQwtTarGz{} and read \ReadMePatches{} to check minor bug fixes and
  upgrades.
\item
  \begin{verbatim}
cd PyQwt-3.8
  \end{verbatim}
  {}
\item
  Edit the file \file{setup_cfg_posix} as explained in its comments.
  \strong{Initialize}{} \var{sip_file_dirs}{} \strong{correctly!}.
  \verbatiminput{setup_cfg_posix}
\item
  The environment variable \envvar{QTDIR} must be set to the Qt directory
  containing the sub-directories \file{bin}, \file{lib} and \file{include}.
  You may even try to install PyQwt on Qt installations with a non-standard
  directory layout, if you set either \envvar{QTBINDIR} and/or
  \envvar{QTINCDIR} and/or \envvar{QTLIBDIR}.
\item
  If you use Qt-2.3.x, the environment variable \envvar{TMAKESPEC} must be set
  to the directory containing the file \file{tmake.conf} for your compiler.
  If you use Qt-3.x, you may have to set the environment variable
  \envvar{QMAKESPEC}.
\item
  Run
  \begin{verbatim}
python setup.py build
  \end{verbatim}
  to build the PyQwt extension.
\item
  Play with the \file{*.py} files in the examples directory.
  If everything works, you are ready to install.
\item
  After the removal of \emph{all} files of previous installations of PyQwt
  in your Python library tree, you have the choice between installing:
  \begin{verbatim}
python setup.py install
  \end{verbatim}
  or anticipating future removal of your new PyQwt installation:
  \begin{itemize}
  \item
    in case of a RPM based distribution, running
\begin{verbatim}
python setup.py bdist_rpm
\end{verbatim}
    makes RPM packages in the sub-directory \file{dist}.
    Uninstall an obsolete RPM package of PyQwt by running
    \begin{verbatim}
rpm -e PyQwt
    \end{verbatim}
    before installing a new RPM package with a command like
    \begin{verbatim}
rpm -Uvh dist/PyQwt*.i586.rpm
    \end{verbatim}
    {}
  \item
    in other cases, running
    \begin{verbatim}
python setup.py bdist_dumb
    \end{verbatim}
    makes a "dumb" distribution: an archive to be unpacked under /.
    If you keep the archive, you you have a trace of the installed files.
    Debian users may be able to convert the archive to a Debian package
    with \program{alien}.
  \end{itemize}
\end{enumerate}

\begin{notice}[note]
  The PyQtDistutils scripts will guide you in setting the environment variable
  \envvar{QTDIR} (and \envvar{TMAKEPATH} or \envvar{QMAKESPEC}, if required).
\end{notice}

\begin{notice}[note]
  If you get a message:
  \begin{verbatim}
sip: Unable to find file "qtmod.sip"
error: command '/usr/bin/sip' failed with exit status 1
  \end{verbatim}
  the initialization of \var{sip_file_dirs} is wrong.
\end{notice}

\begin{notice}[note]
  If you are desperate, run:
  \begin{verbatim}
make force
  \end{verbatim}
  and mail the file \file{LOG.force} to \me.
\end{notice}

\begin{notice}[note]
  Look at the Makefile for examples how to use
  \begin{verbatim}
python setup.py command --options
  \end{verbatim}
  For example, the command:
  \begin{verbatim}
python setup.py run_sip --sip-file-dirs=../PyQt-x11-gpl-3.8/sip build
  \end{verbatim}
  tries to build the extension, overruling any initialization of
  \var{sip_file_dirs} in the configuration file.
\end{notice}

\section{Source Installation on Windows \label{win-source-install}}

Building PyQwt from source on Windows with Microsoft Visual C++ uses the
Python Distutils tools: it produces a PyQwt*.exe installer at no cost!
I have only experience with building such an installer for \QtNcExe{},
but I know from feedback that the procedure for Qt-3.x.x is similar.

\subsection{Prerequisites for building PyQwt on Windows
  \label{win-source-prereq}}

\begin{notice}[warning]
This documentation is for the future PyQwt-3.8. If you are using \PyQwtExeBB,
you will need \PythonExeBB{}, \NumericExeBB{}, \QtNcExe{}.
\end{notice}

You have to install recent releases of Python, Numeric, and Qt-Non-Commercial,
by running:
\begin{enumerate}
  \item
    \PythonExeBC{}
  \item
    \NumericExeBC{}
  \item
    \QtNcExe{}.
\end{enumerate}

\begin{notice}[note]
Read \ReadMePatches{} to check out bug fixes and upgrades.
\end{notice}

\begin{notice}[note]
Presumably, there is also a Qt-Educational installer, but I have no
idea how Qt-Commercial is delivered.
\end{notice}

You have to make sure that your installation of sip and PyQt allows you to
build PyQwt:
\begin{enumerate}
  \item
    If you are going to use PyQt-win-nc-3.8 (tested) or PyQt-win-edu-3.8
    (untested), remove \emph{all} files belonging to other releases of sip
    and PyQt and do a custom install to select the SIP tools.
    In this case you can skip rebuilding of sip and PyQt.

  \item
    If you have installed PyQt\&sip-3.8-Commercial and did not delete the
    build trees, you can \emph{probably} skip rebuilding of sip and PyQt.

  \item
    In all other cases, you have to rebuild sip and PyQt to make sure that
    all necessary are present for the build of PyQwt.
\end{enumerate}

\begin{notice}[warning]
  My initial build of PyQwt for PyQt-3.6 failed, because the installer did
  not overwrite the files \file{C:/Python22/include/sip.h} and
  \file{C:/Python22/include/sipQt.h} belonging to a previous PyQt
  installation.
\end{notice}


\subsubsection{Windows build procedure for SIP and PyQt
  \label{win-sip-pyqt-build}}

\begin{notice}[note]
The build procedure for PyQt\&sip-Commercial is very similar to the build
procedure for PyQt\&sip-NonCommercial.
Of course, you should have installed Qt-Commercial and you have to start from
the appropiate zip files in the procedure below.
\end{notice}

The build procedure for sip and PyQt starting from
\SipNcZip{} and \PyQtNcZip{} is:
\begin{enumerate}
  \item
    Unzip \SipNcZip{} into for instance \file{C:/MyStuff}.
    Read the \file{README} in the sip source tree and \file{win-sip.bat} batch
    file in the PyQwt sources.
    Use the batch file to install sip (the generated Makefile in the siplib
    folder to install two necessary header files are buggy).
  \item
    Unzip \PyQtNcZip{} into for instance \file{C:/MyStuff}. 
    Read the \file{README} in the PyQt source tree and the \file{win-pyqt.bat}
    batch file in the PyQwt sources.
    Use the batch file to install PyQt. It builds all of PyQt and does a
    partial install: the documentation and examples are missing, but all the
    necessary library and Python files are there!
\end{enumerate}

\begin{notice}[warning]
  Do not run
  \begin{verbatim}
nmake install
  \end{verbatim}
  because it may delete the necessary export libraries.
\end{notice}

\begin{notice}[warning]
  Most bug reports are related to he presence of
  old executables (\file{sip.exe}),
  old libraries (\file{libqt*.exp}, \file{libqt*.lib}, \file{lib.qt*.pyd},
  \file{libsip.exp}, \file{libsip.lib}, \file{libsip.pyd}),
  old header files (\file{sip.h}, \file{sipQt.h})
  or old Python files (\file{qt*.py}, \file{qt*.pyc}).
\end{notice}

\begin{notice}[warning]
  You may need to complete the file \file{win-pyqt.bat} to add Python wrapper
  files that are not part of PyQt-win-nc-3.8.
\end{notice}

\begin{notice}[note]
  PyQt's link step is long enough for a lunch.
\end{notice}

\begin{notice}[note]
  To build \program{pyuic}, I have copied the source tree of
  the \code{zlib} library to \file{C:/qt/src/3rdparty/zlib}, because
  it is missing from \QtNcExe.
\end{notice}

\subsection{Build on Windows \label{win-build}}

\begin{enumerate}
\item
  Unpack \PyQwtTarGz{} into for instance \file{C:/MyStuff} and read
  \ReadMePatches{} to check out bug fixes and upgrades.
\item
  Run
  \begin{verbatim}
cd PyQwt-3.8
  \end{verbatim}
  {}
\item
  Edit the file \file{setup_cfg_nt} as explained in its comments.
  \strong{Initialize}{} \var{sip_file_dirs}{} \strong{correctly!}.
  \verbatiminput{setup_cfg_nt}
\item
  The commands
  \begin{verbatim}
python setup.py bdist
  \end{verbatim}
  or
  \begin{verbatim}
python setup.py bdist_wininst
  \end{verbatim}
  will make an installer in the sub-directory \file{dist}.
\end{enumerate}

\begin{notice}[note]
  The PyQtDistutils rely on the environment variables \envvar{QTDIR},
  \envvar{TMAKESPEC} for Qt-2.x and \envvar{QMAKESPEC} for Qt-3.x.
\end{notice}

\begin{notice}[note]
  If you get a message:
  \begin{verbatim}
sip: Unable to find file "qtmod.sip"
error: command '/usr/bin/sip' failed with exit status 1
  \end{verbatim}
  the initialization of \var{sip_file_dirs} is wrong.
\end{notice}

\begin{notice}[note]
  If you are desperate, run:
  \begin{verbatim}
python setup.py build --force >log.txt
  \end{verbatim}
  and mail the file \file{log.txt} to \me.
\end{notice}


\chapter{PyQwt Module Reference \label{reference}}

The reference should be used in conjunction with the
\ulink{Qwt manual}{qwt/index.html} on your hard disk or \Qwt{} on the web.
Only the differences specific to the Python bindings are documented here.

In this chapter, \emph{is not yet implemented} implies that the feature can
be easily implemented if needed, \emph{is not implemented} implies that the
feature is not easily implemented, and \emph{is not Pythonic} implies that
the feature will not be implemented because it violates the Python philosophy
(e.g. may use dangling pointers).

If a class is described as being \emph{fully implemented} then all non-private
member functions and all public class variables have been implemented.

Undocumented classes have not yet been implemented or are still experimental.

The classes in the Qwt library have quite a few protected attributes.
They are not yet available in Python (as far as I know, sip wraps
protected member function, but not protected attributes).
I may try to implement protected attributes on demand.

\begin{classdesc}{QwtArrowButton}{*args}
\emph{is fully implemented}.
\end{classdesc}

\begin{classdesc}{QwtAutoScale}{*args}

  \begin{itemize}
  
    \item{adjust}. C++ declaration:
      \begin{verbatim}
void adjust(double minVal, double maxVal, int reset = 0);
void adjust(double* x, int size, int reset = 0);
      \end{verbatim}
      Python usage:
      \begin{verbatim}
autoScale.adjust(minVal, maxVal, reset = 0)
autoScale.adjust(x, reset = 0)
    \end{verbatim}
      The first form takes two floats and a boolean.
      The second form takes a list, tuple or NumPy array of floats and a
      boolean.

  \end{itemize}
\end{classdesc}

\begin{classdesc}{QwtCompass}{*args}
\emph{is fully implemented}.
\end{classdesc}

\begin{classdesc}{QwtCompassMagnetNeedle}{*args}
\emph{is fully implemented}.
\end{classdesc}

\begin{classdesc}{QwtCompassRose}{*args}
\emph{is fully implemented}.
\end{classdesc}

\begin{classdesc}{QwtCompassScaleDraw}{*args}
\emph{is fully implemented}. This class is experimental.
\end{classdesc}

\begin{classdesc}{QwtCompassSimpleRose}{*args}
\emph{is fully implemented}.
\end{classdesc}

\begin{classdesc}{QwtCompassWindArrow}{*args}
\emph{is fully implemented}.
\end{classdesc}

\begin{classdesc}{QwtCurve}{*args}

  \begin{itemize}

    \item{setData}. C++ declaration:
      \begin{verbatim}
void setData(double* x, double* y, int size)
      \end{verbatim}
      Python usage:
      \begin{verbatim}
curve.setData(x, y)
      \end{verbatim}
      \var{x} and \var{y} can be any combination of lists, tuples and Numeric
      arrays of floats.
      All sequences are internally truncated to the length of the shortest
      sequence.

    \item{setRawData}. C++ declaration:
      \begin{verbatim}
      void setRawData(double* x, double* y, int size)
      \end{verbatim}
      \code{setRawData} \emph{is not Pythonic}.

    \item{verifyRange}. C++ declaration:
      \begin{verbatim}
      int verifyRange(int& i1, int& i2);
      \end{verbatim}
      Python usage:
      \begin{verbatim}
      length, first, last = curve.verifyRange(i1, i2)
      \end{verbatim}

  \end{itemize}
\end{classdesc}

\begin{classdesc}{QwtDial}{*args}
\emph{is fully implemented}.
\end{classdesc}

\begin{classdesc}{QwtDialScaleDraw}{*args}
\emph{is fully implemented}.
\end{classdesc}

\begin{classdesc}{QwtDialNeedle}{*args}
\emph{is fully implemented}.
\end{classdesc}

\begin{classdesc}{QwtDialSimpleNeedle}{*args}
\emph{is fully implemented}.
\end{classdesc}

\begin{classdesc}{QwtDiMap}{*args}
\emph{is fully implemented}.
\end{classdesc}

\begin{classdesc}{QwtDblRange}{*args}
\emph{is fully implemented}.
\end{classdesc}

\begin{classdesc}{Qwt}{*args}
\emph{is fully implemented}.
\end{classdesc}

\begin{classdesc}{QwtGrid}{*args}
\emph{is fully implemented}.
\end{classdesc}

\begin{classdesc}{QwtKnob}{*args}
\emph{is fully implemented}. See QwtSliderBase.
\end{classdesc}

\begin{classdesc}{QwtLegend}{*args}
\emph{is fully implemented}.
\end{classdesc}

\begin{classdesc}{QwtLegendItem}{*args}
\emph{is fully implemented}.
\end{classdesc}

\begin{classdesc}{QwtMarker}{*args}
\emph{is fully implemented}.
\end{classdesc}

\begin{classdesc}{QwtPaintBuffer}{*args}
\emph{is fully implemented}.
\end{classdesc}

\begin{classdesc}{QwtPainter}{*args}
\emph{is fully implemented}.
\end{classdesc}

\begin{classdesc}{QwtPlainText}{*args}
\emph{is fully implemented}.
\end{classdesc}

\begin{classdesc}{QwtPlot}{*args}

  \begin{itemize}

    \item{print}. C++ declaration:
      \begin{verbatim}
void print(QPrinter &, const QwtPlotPrintFilter &);
void print(QPainter *, const QRect &, const QwtPlotPrintFilter &);
      \end{verbatim}
      \code{print} has been renamed to \code{printPlot} in Python.

    \item{curveKeys}. C++ declaration:
      \begin{verbatim}
QArray<long> curveKeys() const;
      \end{verbatim}
      \code{curveKeys} returns a list containing the keys of all curves.

    \item{closestCurve}. C++ declaration:
      \begin{verbatim}
long closestCurve(int xPos, int yPos, int& distance) const;
long closestCurve(int xPos, int yPos, int& distance,
                  double& xVal, double& yVal, int& index) const;
      \end{verbatim}
      Python usage:
      \begin{verbatim}
key, distance, xVal, yVal, index = plot.closestCurve(xPos, yPos)
      \end{verbatim}

    \item{setCurveData}. C++ declaration:
      \begin{verbatim}
bool setCurveData(long key, double *x, double *y, int size);
      \end{verbatim}
      Python usage:
      \begin{verbatim}
plot.setCurveData(key, x, y)
      \end{verbatim}
      \code{key} is an index to the curve and \code{x} and \code{y} are any
      combination of lists, tuples or NumPy arrays of Python floats.
      All sequences are internally truncated to the length of the shortest
      sequence.

    \item{setCurveRawData}. C++ declaration:
      \begin{verbatim}
bool setCurveRawData(long key, double* x, double* y, int size);
      \end{verbatim}
      \code{setCurveRawData} is not Pythonic.

    \item{axisMargins}. C++ declaration:
      \begin{verbatim}
bool axisMargins(int axis, double& lowMargin, double& highMargin) const;
      \end{verbatim}
      Python usage:
      \begin{verbatim}
valid, lowMargin, highMargin = plot.axisMargins(axis)
      \end{verbatim}
      If the axis is valid, valid=1. If the axis is invalid, valid=0.

    \item{axisLabelFormat}. C++ declaration:
      \begin{verbatim}
void axisLabelFormat(int axis, char& format, int& precision, int& width) const;
      \end{verbatim}
      Python usage:
      \begin{verbatim}
format, precision, width = plot.axisLabelFormat(axis)
      \end{verbatim}

    \item{closestMarker}. C++ declaration:
      \begin{verbatim}
long closestMarker(int xPos, int yPos, int& distance) const;
      \end{verbatim}
      Python usage:
      \begin{verbatim}
key, distance = plot.closestMarker(xPos, yPos)
      \end{verbatim}
      \code{closestMarker} takes the x- and y-coordinates of a point and
      returns a tuple containing the key of the closest marker and the distance
      between the closest marker and the point.

    \item{markerKeys}. C++ declaration:
      \begin{verbatim}
QArray<long> markerKeys() const;
      \end{verbatim}
      \code{markerKeys} returns a list containing the keys of all markers.

    \item{markerPos}. C++ declaration:
      \begin{verbatim}
void markerPos(long key, double& x, double& y) const;
      \end{verbatim}
      Python usage:
      \begin{verbatim}
x, y = plot.markerPos(key)
      \end{verbatim}
      \code{markerPos} takes the key of a marker and returns a tuple containing
      the x- and y-coordinates.

  \end{itemize}
\end{classdesc}

\begin{classdesc}{QwtPlotCanvas}{*args}
\emph{is fully implemented}.
\end{classdesc}

\begin{classdesc}{QwtPlotCurve}{*args}
\emph{is fully implemented}.
\end{classdesc}

\begin{classdesc}{QwtPlotGrid}{*args}
\emph{is fully implemented}.
\end{classdesc}

\begin{classdesc}{QwtPlotItem}{*args}
\emph{is fully implemented}.
\end{classdesc}

\begin{classdesc}{QwtPlotLayout}{*args}
\emph{is fully implemented}.
\end{classdesc}

\begin{classdesc}{QwtPlotMappedItem}{*args}
\emph{is fully implemented}.
\end{classdesc}

\begin{classdesc}{QwtPlotPrintFilter}{*args}
\emph{is fully implemented}.
\end{classdesc}

\begin{classdesc}{QwtRect}{*args}
\emph{is fully implemented}.
\end{classdesc}

\begin{classdesc}{QwtRichText}{*args}
\emph{is fully implemented}.
\end{classdesc}

\begin{classdesc}{QwtScale}{*args}

  \begin{itemize}

    \item{labelFormat}. C++ declaration:
      \begin{verbatim}
void labelFormat(char& format, int& precision, int& width) const;
      \end{verbatim}
      Python usage:
      \begin{verbatim}
format, precision, width = scale.labelFormat()
      \end{verbatim}

    \item{minBorderDist}. C++ declaration:
      \begin{verbatim}
void minBorderDist(int& start, int& end) const;
      \end{verbatim}
      Python usage:
      \begin{verbatim}
start, end = scale.minBorderDist()
      \end{verbatim}

  \end{itemize}
\end{classdesc}

\begin{classdesc}{QwtScaleDiv}{*args}
\emph{is fully implemented}.
\end{classdesc}

\begin{classdesc}{QwtScaleDraw}{*args}
\emph{is fully implemented}.
\end{classdesc}

\begin{classdesc}{QwtScaleIf}{*args}
\emph{is fully implemented}.
\end{classdesc}

\begin{classdesc}{QwtSlider}{*args}
\emph{is fully implemented}. See QwtSliderBase.
\end{classdesc}

\begin{classdesc}{QwtSliderBase}{*args}
\emph{is fully implemented}.

  \begin{itemize}

    \item{getScrollMode}. C++ declaration:
      \begin{verbatim}
virtual void getScrollMode(const QPoint& point, int& mode, int& direction) = 0;
      \end{verbatim}
      Python usage:
      \begin{verbatim}
mode, direction = sliderBase.getScrollMode(point)
      \end{verbatim}
      See QwtKnob, QwtSlider and QwtWheel.

  \end{itemize}
\end{classdesc}

\begin{classdesc}{QwtSpline}{*args}
  
  \begin{itemize}

    \item{recalc}. C++ declaration:
      \begin{verbatim}
int recalc(double* x, double* y, int n, bool periodic = 0);
      \end{verbatim}
      Python usage:
      \begin{verbatim}
errorCode = spline.recalc(x, y, periodic)
      \end{verbatim}
      \var{x} and \var{y} are a sequence of floats and \var{periodic} is a
      boolean.
      The sequences can be a list, a tuple or a NumPy array and all sequences
      are internally truncated to the length of the shortest sequence.

    \item{copyValues}. C++ declaration:
      \begin{verbatim}
void copyValues(int enable = 1);
      \end{verbatim}
      copyValues is not Pythonic.

  \end{itemize}
\end{classdesc}
 
\begin{classdesc}{QwtSymbol}{*args}
\emph{is fully implemented}.
\end{classdesc}

\begin{classdesc}{QwtText}{*args}
\emph{is fully implemented}.
\end{classdesc}

\begin{classdesc}{QwtThermo}{*args}
\emph{is fully implemented}.
\end{classdesc}

\begin{classdesc}{QwtWheel}{*args}
\emph{is fully implemented}. See QwtSliderBase.
\end{classdesc}

%begin{latexonly}
\renewcommand{\indexname}{Index}
%end{latexonly}
\input{\jobname.ind}

\end{document}
